\begin{otherlanguage}{french}
\begin{abstract}
\tiny{		Cette thèse se concentre sur deux questions de la théorie de modèles cinétiques.
	
	Dans la première partie on étudie la connexion entre les modèles cinétiques avec la statistique de Fermi-Dirac et la dynamique macroscopique de fluide. On obtient ces limites macroscopiques lorsque la fluide est suffisamment dense pour que les particules fassent beaucoup de collisions pour une unité de temps. Cette situation est décrite par un petit paramètre $\ve$, appelé le nombre de  Knudsen, qui représente la rapport du libre parcours moyen des particules entre les collisions et de la longueur caractéristique du flot.
	
	On dérive les limites formelles; pour ce faire, on introduit un scaling pour l'équation cinétique de la forme 
	\begin{equation*}
	\partial_t F_\varepsilon +v\cdot \nabla_xF_\varepsilon =\frac{1}{\varepsilon }C(F_\varepsilon ).
	\end{equation*}
	Ici $F_\varepsilon$ est une fonction non-négative représentante la densité des particules avec la position  $x$ et vitesse $v$ dans l'espace de phase  $\mathbb R^3_x\times\mathbb R^3_v$ au temps $t$. L'interaction des particules via des collisions est donnée par l'opérateur  $C(F)$; cet opérateur agit seulement sur la variable  $v$ et il est non-linéaire dans le cas général. 
	
	On base cette connexion entre la dynamique cinétique et la dynamique macroscopique sur les lois de conservation et les relations d'entropie impliquant que les états d’équilibre sont des distributions de Fermi-Dirac (i.e. de la forme
	$1/(1+\exp(c_1+c_2|v-u|^2)$).
	
	Dans le premier chapitre on établit que les moments et les paramètres des distributions de Fermi-Dirac sont liées par un difféomorphisme. En plus, pour une large classe des opérateurs de collisions on donne les conditions sous lesquelles on peut dériver formellement les équations de Euler généralisées à partir de l'équation de Boltzmann. Ces conditions sont liées au théorème H et elles supposent une  convergence formellement consistante de moments dynamiques de fluide et d'entropie de l'équation cinétique. On discute également si ces équations d'Euler sont bien posées en utilisant le critère d'hyperbolicité de Godunov.
	
	Dans le deuxième chapitre on utilise le développement de  Chap\-man-Enskog pour étudier la relation entre les solutions des équations cinétiques avec la statistique de Fermi-Dirac et les  solutions des équations de Navier-Stokes  compressibles  pour une forme spécifique de l'opérateur de collision $C$.
	
	On établit les propriétés analytiques d'opérateur de collision linéarisé; en particulier, on démontre que sous certaines hypothèses sur le noyau de collision  l'opérateur de collision linéarisé satisfait l'alternative de Fredholm. On décrit l'approche générale permettant de réutiliser des résultat existants pour le cas Maxwellien. On construit les solutions approchées d'ordre deux de l’équation cinétique sur la base des solutions d'équations de Navier-Stokes avec une forme particulière de dissipation, viscosité et flot de chaleur.
	
	
	Dans le troisième chapitre on étend les résultats obtenus dans les chapitres précédents par établir la forme limite des équations de dynamique de fluide incompressible. On introduit les différents échelles  de temps et d'espace pour les équations cinétiques avec la statistique de Fermi-Dirac pour le même opérateur de collision que dans le deuxième chapitre. Sous les hypothèses plus fortes et la convergence formelles des moments de dynamique de fluide, à l'aide de méthode de développement des moments, on obtient comme la limite formelle les équations de dynamique de fluide incompressible.

Dans le quatrième chapitre on considère l'équation de Boltzmann linéarisée près d'une fonction Maxwellienne globale. Une Maxwellienne globale est une fonction Maxwellienne locale qui satisfait en même temps l'équation de transport libre. On établit les propriétés analytiques de l’opérateur de collision linéarisé, notamment les conditions suffisantes pour que cet opérateur soit borné. Puis on démontre l'existence de solutions de l'équation de Boltzmann  linéarisée pour un problème à valeur initiale et pour les problèmes à valeur à bord. Pour ces solutions on montre l'existence de limites au temps grand.  Ces résultats nous permettent d'introduire l'opérateur de scattering qu l'on peut comprendre comme \enquote{évolution de la fonction de densité modulo transport libre}. Le résultat clé de ce chapitre est que cet opérateur de scattering est borné et qu'il possède un gap spectrale dans un espace de Hilbert à poids spécifique.
}
\end{abstract}
\end{otherlanguage}