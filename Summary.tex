\begin{abstract}


\tiny{This thesis concentrates on two principal questions arising in the kinetic theory.

In the first part we study the connection between kinetic models with Fermi-Dirac statistics and macroscopic fluid dynamics. We obtain these macroscopic limits when the fluid is dense enough that particles undergo many collisions per unit of time. This situation is described via a small parameter
$\varepsilon$, called the Knudsen number, that represents the ratio of mean free path of particles between collisions to some characteristic length of the flow.
We derive formal limits; in order to do that, we introduce a scaling for standard kinetic equation of the form
\begin{equation*}%\label{eq:scaled}
\partial_t F_\varepsilon +v\cdot \nabla_xF_\varepsilon =\frac{1}{\varepsilon }C(F_\varepsilon ).
\end{equation*}
Here $F_\varepsilon$ is a non-negative function representing the density of particles with position $x$ and velocity $v$ in the single-particle phase space $\mathbb R^3_x\times\mathbb R^3_v$ at time $t$. The interaction of particles through collisions is given by the operator $C(F)$; this operator acts only on variable $v$ and is non-linear in the general case. 

We base the connection between kinetic and macroscopic dynamics  on the 
 conservation properties and entropy relations
implying that the equilibria are Fermi-Dirac (i.e. of the form
$1/(1+\exp(c_1+c_2|v-u|^2)$) distributions.

In the first chapter we establish that moments and parameters of Fermi-Dirac distributions are related by a diffeomorphism. Also, for a very large class of collision operators we give the conditions that allow us to formally derive the generalised Euler equations from the Boltzmann equation.
  These conditions are related to the H-theorem and assume a formally consistent convergence for fluid dynamical moments and entropy of the kinetic equation. We also discuss the well-posedness of the obtained Euler equations by using Godunov's criterion of hyperbolicity. 

In the second chapter we use the Chapman-Enskog expansion to study the connection between the solutions of kinetic equations with Fermi-Dirac statistics and the solutions of the compressible Navier-Stokes equations for a specific form of the collision operator $C$.

We establish the analytic properties of the linearised collision operator; in particular, we prove that under certain hypothesis on the collision kernel the linearised collision operator satisfies the Fredholm alternative. We describe a general approach allowing to reuse the existing results from Maxwellian case. We build approximate solutions of order two of the scaled kinetic equation by using the solution of the Navier-Stokes equations with a particular form of viscosity and heat flow.

In the third chapter we extend the results obtained in the previous two chapters    by establishing the limiting form of the fluid dynamic equations in the incompressible case. We introduce several scalings for the kinetic equations with the Fermi-Dirac statistics with the same collision operator as in the second chapter. Under   stronger assumptions and a formally consistent convergence for the fluid dynamical moments we can formally derive the limiting fluid dynamic equations with the help of the moment method expansion. 

In the fourth chapter we consider the Boltzmann equation linearised about a global Maxwellian; global Maxwellian distribution function are local Maxwellian functions satisfying the free transport equation. We establish analytical properties on the linearised collision operator, most notably, sufficient conditions for its continuity. Then we prove the existence of solutions of the linearised Boltzmann equation for an initial value problem and for boundary value problems, and, moreover, prove the existence of limits for large time of these solutions. These results allow us to introduce the scattering operator, which can be understood as \enquote{evolution of the density function modulo free transport}. The key result of this chapter is that this scattering operator is bounded and it has a spectral gap in a weighted Hilbert space.}
\end{abstract}
